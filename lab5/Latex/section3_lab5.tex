\documentclass[notitlepage, a4paper, 11pt]{article}

\usepackage{geometry}
\geometry{
	a4paper,
	total={170mm,257mm},
	left=20mm,
	top=20mm,
}

\usepackage{ gensymb }
\usepackage{wrapfig}
\usepackage{xcolor}
\usepackage{graphicx}
\usepackage{amsmath}
\usepackage{listings}
\usepackage{xcolor}
\usepackage{minted}
\usepackage{tikz}
\usepackage[european]{circuitikz}
\usepackage{caption}
\usepackage{subcaption}
\usepackage{hyperref}
\hypersetup{
	pdfborder = false,
	colorlinks=true,
	linkcolor=black,
	filecolor=black,      
	urlcolor=blue,
	pdftitle={Lab report 5},
	pdfpagemode=FullScreen,
}
\title{Two-port Networks\\
	\large Laboratory V}
\author{Patrycja Nazim, Adrian Król, Gabriel Ćwiek, Kamil Chaj}
\date{}

\begin{document}
	\maketitle
	\section{Goal of the exercise}
	The aim of this exercise is to familiarize with experimental methods of determining two-port network parameters by measuring output and input voltages and currents, and then compare theoretical and experimental results
	\section{Two-port networks}
	\section{Course of measurements}
	\begin{figure}[H]
		\centering
				\begin{subfigure}{0.45\textwidth}
			\centering
			\begin{circuitikz}[scale = 0.8, transform shape]
				\ctikzset{bipoles/length = 10mm}
				\node [bnc, scale=2] at (0, 0) (CON1) {};
				\node [bnc, scale=2, anchor = zero, xscale=-1] at (8, 0) (CON2) {};
				\node [ground] at (CON1.shield) {};
				\node [ground] at (CON2.shield) {};
				\draw (CON1.hot) -- (1, 0)
				to[R, l=$RS_{21}$, *-*] (3, 0)
				to[C, l2=$C_{21}$ and 10nF, l2 halign=c, l2 valign=b](5, 0)
				to[R, l=$RS_{22}$, *-*](7, 0) -- (CON2.hot);
				\draw (3,0)
				to[R, l2=$R_{21}$ and $1.1k\Omega$, l2 halign=c, l2 valign=c] (3,-2)
				node[tlground]{};
				\draw (5,0)
				to[R, l2=$R_{22}$ and $4.7k\Omega$, l2 halign=c, l2 valign=c] (5,-2)
				node[tlground]{};
				\draw (1, 0) -- (1, 1)
				to[nopb, l=\small JP21](3, 1) -- (3, 0);
				\draw (5, 0) -- (5, 1)
				to[nopb, l=\small JP22](7, 1) -- (7, 0);
				\draw (1, 0)
				to[nopb, l=\small JP23](1, -2)
				node[tlground] {};
				\draw (7, 0)
				to[nopb, l=\small JP24](7, -2)
				node[tlground] {};
			\end{circuitikz}
			\caption{Circuit no. 2}
			\label{subfig.circuit-2}
		\end{subfigure}
		\hfill
		\begin{subfigure}{0.45\textwidth}
			\centering
			\begin{circuitikz}[scale = 0.8, transform shape]
				\ctikzset{bipoles/length = 10mm}
				\node [bnc, scale=2] at (0, 0) (CON1) {};
				\node [bnc, scale=2, anchor = zero, xscale=-1] at (8, 0) (CON2) {};
				\node [ground] at (CON1.shield) {};
				\node [ground] at (CON2.shield) {};
				\draw (CON1.hot) -- (1, 0)
				to[R, l=$RS_{31}$, *-*] (3, 0)
				to[C, l2=$C_{31}$ and 22nF, l2 halign=c, l2 valign=b](4, 0)
				to[C, l2=$C_{32}$ and 22nF, l2 halign=c, l2 valign=b](5, 0)
				to[R, l=$RS_{32}$, *-*](7, 0) -- (CON2.hot);
				\draw (4,0)
				to[R, *-, l2=$R_{31}$ and $1k\Omega$, l2 halign=c, l2 valign=c] (4,-2)
				node[tlground]{};
				\draw (1, 0) -- (1, 1)
				to[nopb, l=\small JP31](3, 1) -- (3, 0);
				\draw (5, 0) -- (5, 1)
				to[nopb, l=\small JP32](7, 1) -- (7, 0);
				\draw (1, 0)
				to[nopb, l=\small JP33](1, -2)
				node[tlground] {};
				\draw (7, 0)
				to[nopb, l=\small JP34](7, -2)
				node[tlground] {};
			\end{circuitikz}
			\caption{Circuit no. 3}
			\label{subfig.circuit-3}
		\end{subfigure}
		\caption{Measured circuits}
		\label{fig.circuits}
	\end{figure}
	\section{Theoretical calculations}
	
		\begin{figure}[H]
			\centering
			\begin{circuitikz}[scale = 0.8, transform shape]
				\draw (0, 0) 
				to[sI=$i_1$] (0, 3) -- (2, 3)
				to[R=$R_{21}$] (2, 0) -- (0, 0);
				\draw (2, 3)
				to[C=$C_{21}$] (4, 3)
				to[R=$R_{22}$] (4, 0) -- (2, 0);
				\draw (4, 0) -- (6, 0)
				to[sI=$i_2$] (6, 3) -- (4, 3);
				\node [rground] at (3, 0) {};
				\draw [black, thick, dashed] (-0.2, 3.2) rectangle (2.2, 2.8);
				\draw [black, thick, dashed] (3.8, 3.2) rectangle (6.2, 2.8);
				\node [above] at (1, 3.2) {$v_1$};
				\node [above] at (5, 3.2) {$v_2$};
			\end{circuitikz}
			\caption{Simplified circuit no. 2}
			\label{fig:simplified-circuit-2}
		\end{figure}
		
		\begin{figure}[H]
			\centering
			\begin{circuitikz}[scale = 0.8, transform shape]
				\draw (0, 0) 
				to[sV=$v_1$] (0, 3)
				to[C=$C_{31}$] (3, 3)
				to[C=$C_{32}$] (6, 3);
				\draw (3, 3)
				to[R=$R_{31}$] (3, 0) -- (0, 0);
				\draw (3, 0) -- (6, 0)
				to[sV_=$v_2$] (6, 3);
				\draw[->]   (1.3,2) arc(110:-110:5mm) node[midway, left] {$i_1$};
				\draw[->]   (4.7,2) arc(-110:110:-5mm) node[midway, right] {$i_2$};
			\end{circuitikz}
			\caption{Simplified circuit no. 3}
			\label{fig:simplified-circuit-3}
		\end{figure}
	
	\section{Comparison}
	\section{Conclusions}
	
	
	
\end{document}
