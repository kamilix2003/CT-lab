\documentclass[notitlepage, a4paper, 11pt]{article}

\usepackage{geometry}
\geometry{
	a4paper,
	total={170mm,257mm},
	left=20mm,
	top=20mm,
}

\usepackage{gensymb}
\usepackage{wrapfig}
\usepackage{xcolor}
\usepackage{graphicx}
\usepackage{amsmath}
\usepackage{listings}
\usepackage{xcolor}
\usepackage{minted}
\usepackage{tikz}
\usepackage[european resistors]{circuitikz}
\usepackage{caption}
\usepackage{subcaption}
\usepackage{hyperref}
\hypersetup{
	pdfborder = false,
	colorlinks=true,
	linkcolor=black,
	filecolor=black,      
	urlcolor=blue,
	pdftitle={Overleaf Example},
	pdfpagemode=FullScreen,
}
\title{Laplace Transform\\
	\large Laboratory III}
\author{Patrycja Nazim, Adrian Król, Gabriel Ćwiek, Kamil Chaj}
\date{}

\begin{document}
	\maketitle
	\section{Goal of the exercise}
	\section{Laplace Transform}
	\section{Course of measurements}
		
		\begin{figure}[H]
		\centering
		\begin{subfigure}{0.95\textwidth}
			\centering
			\begin{circuitikz}
				\ctikzset{bipoles/oscope/width=1.5}
				\ctikzset{bipoles/oscope/height=1}
				\draw [black, thick] (0, 0) rectangle (3, -2);
				\node [black, below] at (1.5, -2) {RC circuits};
				\draw [black, thick] (-3, 0) rectangle (-1, -1.5);
				\node [black, above] at (-2, 0) {\small Function Generator};
				\draw (1.5, 1.3) node[oscopeshape](O) {Oscilloscope};
				\node [bnc] at (-1.3, -0.75) (FG) {};
				\node [bnc, font=\tiny, xscale=-1, anchor=zero] at (1, -0.75) (CON1) {};
				\node [bnc, font=\tiny, rotate=90, anchor=zero, label position=45] at (2, -0.75) (CON2) {};
				\node [below, font=\tiny] at (1, -0.85) {CON2};
				\node [below, font=\tiny] at (2, -0.85) {CON1};
				\draw (O.in 1) node[bnc, anchor=zero, rotate=-90](IN1) {};
				\draw (O.in 2) node[bnc, anchor=zero, rotate=-90](IN2) {};
				\draw (FG.hot) to[short, -*] (-0.4, -0.75) -- (-0.4, 0.4) -- (1.08, 0.4) to (IN1.hot);
				\draw (CON1.hot) to[short] (-0.4, -0.75);
				\draw (CON2.hot) to[short] (2, 0.4) -- (1.92, 0.4) -- (IN2.hot);
			\end{circuitikz}
			\caption{Measurement setup}
		\end{subfigure}
		\begin{subfigure}{0.45\textwidth}
			\centering
			\begin{circuitikz}[scale = 0.7, transform shape]
				\draw (0,0) node[bnc](B1) {CON11}
				to[R, l=$R_{11}$, a=1.5k$\Omega$] (3,0)
				to[C, l2=$C_{11}$ and 47nF, l2 halign=c, l2 valign=c] (3,-2)
				node[ground] {}
				;
				\draw (3,0) 
				to[short] (4.5,0)
				node[bnc, xscale=-1](B2){\scalebox{-1}[1]{CON12}}
				;
				\draw node[ground] at (B1.shield) {};
				\draw node[ground] at (B2.shield) {};
			\end{circuitikz}
			\caption{Circuit A}
			\label{fig:Circuit A}
		\end{subfigure}
		\begin{subfigure}{0.45\textwidth}
			\centering
			\begin{circuitikz}[scale = 0.7, transform shape]
				\draw (0,0) node[bnc](B1) {CON21}
				to[C, l=$C_{21}$, a=10nF] (3,0)
				to[R, l2=$R_{21}$ and 1.5k$\Omega$, l2 halign=c, l2 valign=c] (3,-2)
				node[ground] {}
				;
				\draw (3,0) 
				to[short] (4.5,0)
				node[bnc, xscale=-1](B2){\scalebox{-1}[1]{CON21}}
				;
				\draw node[ground] at (B1.shield) {};
				\draw node[ground] at (B2.shield) {};
			\end{circuitikz}
			\caption{Circuit B}
			\label{fig:Circuit B}
		\end{subfigure}
		\caption{RC Circuits}
		\label{fig: Circuit}
	\end{figure}
	
	\begin{figure}[H]
		\centering
		\begin{subfigure}{0.45\textwidth}
			\centering
			\begin{circuitikz}[scale = 0.7, transform shape]
				\node [bnc] at (0,0) (CON41) {};
				\draw (CON41.hot) to[short, -*]
				(1,0) -- (3,0) to[nopb, l_=JP43] (4,0);
				\draw (1,0) -- (1,-2) to[R, l2=$R_{42}$ and $1.1k\Omega$, l2 halign=c, l2 valign=b] (3,-2)
				to[nopb, l_=JP42] (4,-2) -- (4, 0);
				\draw (1,-2) -- (1,-4) to[R, l2=$R_{41}$ and $3.3k\Omega$, l2 halign=c, l2 valign=b] (3,-4)
				to[nopb, l_=JP41] (4,-4) -- (4, -2);
				\draw (4,0) to[short, -*] (5,0)
				to[C, l2=$C_{41}$ and 33nF, l2 halign=c, l2 valign=c] (5,-2) 
				to[L, l2=$L_{41}$ and 10mH, l2 halign=c, l2 valign=c] (5,-4) node[ground] {};
				\draw (5, 0) to (6, 0) node[bnc, xscale=-1, anchor=zero](CON42){};
			\end{circuitikz}
			\caption{Measured Circuit}
		\end{subfigure}
		
		\begin{subfigure}{0.45\textwidth}
			\centering
			\begin{circuitikz}[scale = 0.8, transform shape]
				\ctikzset{bipoles/oscope/width=1.5}
				\ctikzset{bipoles/oscope/height=1}
				\draw [black, thick] (0, 0) rectangle (3, -3);
				\node [black, below] at (1.5, -3) {RC circuits};
				\draw [black, thick] (-3, 0) rectangle (-1, -1.5);
				\node [black, above] at (-2, 0) {\small Function Generator};
				\draw (4.5, 0.6) node[oscopeshape](O) {Oscilloscope};
				\node [bnc] at (-1.3, -0.75) (FG) {};
				\node [bnc, font=\tiny, xscale=-1, anchor=zero] at (0.5, -0.5) (CON1) {\ctikzflipx{CON1}};
				\node [bnc, font=\tiny] at (2.65, -0.5) (CON2) {CON2};
				\draw (CON2.left) to[short, -*] (1.5, -0.5);
				\draw (CON1.left) to[short, -*] (1.5, -0.5)
				to[C, bipoles/length=7mm] (1.5, -1.25)
				to[L, bipoles/length=10mm] (1.5, -2.5) node[tlground, scale=0.8]{};
				\draw (O.in 1) node[bnc, anchor=zero, rotate=-90](IN1) {};
				\draw (O.in 2) node[bnc, anchor=zero, rotate=-90](IN2) {};
				\draw (FG.hot) -- (-0.5, -0.75) -- (-0.5, -0.5) -- (CON1.hot);
				\draw (CON2.hot) -- (4.08, -0.5) -- (IN1.hot);
				\draw [black, ->](IN2.hot) -- (4.92, -1.35) -- (1.5, -1.35);
				\node [below] at (4.52, -1.3){probe};
			\end{circuitikz}
			\caption{Measurements setup}
		\end{subfigure}
		\caption{RLC circuit}
	\end{figure}

	\section{Theoretical calculations}
	\section{Comparison}
	\section{Conclusions}
\end{document}