\documentclass[notitlepage, a4paper, 11pt]{article}

\usepackage{geometry}
\geometry{
	a4paper,
	total={170mm,257mm},
	left=20mm,
	top=20mm,
}

\usepackage{graphicx}
\usepackage{amsmath}
\usepackage{listings}

\usepackage{tikz}
\usepackage[european resistors]{circuitikz}

\title{Nodal Analysis\\
	\large Laboratory I}
\author{Patrycja Nazim, Adrian Król, Kamil Chaj}
\date{}

\begin{document}
	\maketitle
	\section{Aim of exercise}
	The aim of our exercise was to experimentally verify the nodal analysis in RLC circuits. We have achieved it by measuring the voltages on different nodes of the chosen circuits using a dedicated evaluation board and vector voltmeter. The obtained measurement results are compared with analytical calculations.
	
	Apart from the values of potentials in individual nodes of the circuits being measured, we calculated the currents flowing through pointed elements.

	\section{Nodal analysis - method}
	Method which we are going to use to solve this circuit is know as "Nodal Analysis by Inspection". In this method we need to construct 3 matrices: $\mathbf{i}$ - current vector, $\mathbf{u}$ - voltage vector(unknown), $\mathbf{G}$ - conductance matrix with sizes respectively $\mathit{N} \times 1$, $\mathit{N} \times 1$, $\mathit{N}-1 \times \mathit{N}-1$
		\begin{center}
			$\mathbf{Gu=i}$
		\end{center}
		\begin{center}
			\begin{math}
				\begin{bmatrix}
					G_{11} & -G_{12} & -G_{13} \\
					-G_{21} & G_{22} & -G_{23} \\
					-G_{31} & -G_{32} & G_{33} 
				\end{bmatrix}
				\begin{bmatrix}
					U_1 \\
					U_2 \\ 
					U_3
				\end{bmatrix}
				=
				\begin{bmatrix}
					I_1 \\
					I_2 \\
					I_3
				\end{bmatrix}
			\end{math}
		\end{center}
		Where $G_{11}$, $G_{22}$, $G_{33}$ are sums of conductance of each branch connected to the node \newline $G_{12} = G_{21}$, $G_{13} = G_{31}$, $G_{32} = G_{23}$ are sums of conductance of branches between nodes \newline $I_1, I_2, I_3$ are sums of current sources entering or exiting node and $U_1, U_2, U_3$ are unknown voltages that we are trying to find
		\newline\newline
		With simple matrix operation we obtain equation
		$$
		\mathbf{u} = \mathbf{G}^{-1}\mathbf{i}
		$$
		which can be easily calculated
		\newpage
	\section{theoretical calculations}
	We are using Python with library NumPy for all calculation
	\subsection{Circuit A}
		\begin{figure}[!ht] %circuit A
			\begin{center}
			\begin{circuitikz}[scale = 0.75, transform shape]
				\draw 
				(2,-2) to[vsourcesin, l=V] (2,0)
				node[left]{1}to[short, *-](2,2)
				to[R, l=$R_{11}$, a=1k$\Omega$] (6,2)
				node[above]{2}to[short, *-](6,2)
				to[C, l=$C_{11}$, a=100nF] (10,2) -- (10,0)
				node[right]{4}to[short,*-](10,0)
				to[R, l=$R_{12}$, a=1k$\Omega$] (6,0)
				node[above]{3}to[short, -*](6,0)
				to[C, l=$C_{12}$, a=47nF] (2,0)
				;
				\draw 
				(6,0)
				to[L, l=$L_{11}$, a=10mH](6,-2)
				;
				\draw 
				(10,0) to[C, l=$C_{13}$, a=22nF](10,-2)
				;
				\draw (2,-2)
				to[short](10,-2);
				\draw (6,-2)
				node[rground] {} (6,-2);
			\end{circuitikz}
			\caption{theoretical circuit A}
			\label{fig:tA}
		\end{center}
	\end{figure}
	\begin{center}
			\begin{math}
			\begin{bmatrix}
				\frac{1}{Z_{C_12}} + \frac{1}{Z_{C_11}+Z_{R_{11}}} & \frac{-1}{Z_{C_{11}}} & \frac{-1}{Z_{R_{11}}+Z_{C_{11}}} \\
				\frac{-1}{Z_{C_{11}}} & \frac{1}{Z_{R_{12}}}+\frac{1}{Z_{C_{12}}}+\frac{1}{Z_{L_{11}}} & \frac{-1}{Z_{R_{12}}} \\ 
				\frac{-1}{Z_{R_{11}}+Z_{C_{11}}} & \frac{-1}{Z_{R_{12}}} & 	\frac{1}{Z_{C_13}} + \frac{1}{Z_{C_11}+Z_{R_{11}}} + \frac{1}{Z_{R_{12}}}
			\end{bmatrix}
			\begin{bmatrix}
				U_1 \\
				U_3 \\ 
				U_4
			\end{bmatrix}
			=
			\begin{bmatrix}
				\frac{-V}{Z_{R_{11}}}-\frac{1}{Z_{C_11}+Z_{R_{11}}} \\
				\frac{V}{Z_{C_{12}}} \\
				\frac{V}{Z_{R_{12}}+Z_{C_{11}}}
			\end{bmatrix}
		\end{math}
	\end{center}
	\begin{lstlisting}

	\end{lstlisting}	
	\subsection{Circuit B}
		\begin{figure}[!ht] %circuit B
		\begin{center}
			\begin{circuitikz}[scale = 0.75, transform shape]
				\draw (0,0)
				to[vsourcesin, l=V](2,0)
				node[above]{1}to[short,*-,](2,0)
				to[C, l=$C_{22}$, a=22nF](4,0)
				node[right]{3}to[short,-*](4,0)
				;
				\draw (0,3)
				to[R, l=$R_{21}$, a=1k$\Omega$](4,3)
				node[above]{2}to[short, -*](4,3)
				;
				\draw (0,-3)
				to[C, l=$C_{23}$, a=100nF](4,-3)
				node[below]{4}to[short, -*](4,-3)
				;
				\draw 
				(4,3) -- (7,3)
				to[R, l=$R_{22}$, a=1k$\Omega$](7,-3) -- (4,-3)
				to[L, l=$L_{21}$, a=10mH](4,0)
				to[C, l=$C_{21}$, a=47nF](4,3)
				;
				\draw (0,3)
				to[short](0,-3);
				\draw (0,0)
				node[rground, rotate=-90] {} (0,0);
			\end{circuitikz}
			\label{fig:tB}
			\caption{theoretical circuit B}
		\end{center}
	\end{figure}
	\begin{center}
			\begin{math}
			\begin{bmatrix}
				\frac{2}{Z_{R_{21}}} + \frac{1}{Z_{L_{21}}} & \frac{-1}{Z_{C_{21}}} & \frac{-1}{Z_{R_{22}}} \\
				\frac{-1}{Z_{C_{21}}} & \frac{1}{Z_{C_21}} + \frac{1}{Z_{C_{22}}} + \frac{1}{Z_{L_{21}}} &
				\frac{-1}{Z_{L_{21}}} \\
				\frac{-1}{Z_{R_{22}}} & \frac{-1}{Z_{L_{21}}} & \frac{1}{Z_{C_{23}}} + \frac{1}{Z_{L_{21}}} + \frac{1}{Z_{R_{22}}}
			\end{bmatrix}
			\begin{bmatrix}
				U_2 \\
				U_3 \\ 
				U_4
			\end{bmatrix}
			=
			\begin{bmatrix}
				0 \\
				\frac{V}{Z_{C_{22}}} \\
				0
			\end{bmatrix}
		\end{math}
	\end{center}
		\begin{lstlisting}
		import
	\end{lstlisting}
	\newpage
	\section{real measurements}
	\subsection{Circuit A}
			\begin{figure}[!ht] %circuit A
			\begin{center}
				\begin{circuitikz}[scale = 0.75, transform shape]
					\draw 
					(1,0) node[bnc](B){CON2} to[short](2,0)
					node[below]{1}to[short, *-](2,2)
					to[R, l=$R_{11}$, a=1k$\Omega$] (6,2)
					node[above]{2}to[short, *-](6,2)
					to[C, l=$C_{11}$, a=100nF] (10,2) -- (10,0)
					node[left]{4}to[short,*-](10,0)
					to[R, l=$R_{12}$, a=1k$\Omega$] (6,0)
					node[above]{3}to[short, -*](6,0)
					to[C, l=$C_{12}$, a=47nF] (2,0)
					;
					\node[ground] at (B.shield){};
					\draw 
					(6,0)
					to[L, l=$L_{11}$, a=10mH](6,-2)
					to[short] node[ground] {} (6,-2)
					;
					\draw 
					(10,0) to[C, l=$C_{13}$, a=22nF](10,-2)
					to[short] node[ground] {} (10,-2)
					;
				\end{circuitikz}
				\caption{circuit A}
				\label{fig:A}
			\end{center}
		\end{figure}
	
		\begin{table}[!ht] %table circuit A
		\begin{center}
			\resizebox{75mm}{!}{
			\begin{tabular}{|c|c|c|c|}
				\hline
				\multicolumn{4}{|c|}{\textbf{Circuit A:}} \\
				\hline\hline
				Freq [kHz]: & Channel 1 [V]: & Channel 2 [V]: & Angle[°]: \\
				\hline
				\multicolumn{4}{|c|}{Node 1:   } \\
				\hline
				1kHz & 1.117 & 1.115 &   \\
				\hline
				5kHz & 1.122&1.119 &   \\
				\hline
				9kHz & 1.121 & 1.119 &   \\
				\hline
				\multicolumn{4}{|c|}{Node 2:   } \\
				\hline
				1kHz & 1.117 & 0.830 & -19.5 \\
				\hline
				5kHz & 1.122 & 0.338 & 14.0 \\
				\hline
				9kHz & 1.121 & 1.342 & -11.7 \\
				\hline
				\multicolumn{4}{|c|}{Node 3:   } \\
				\hline
				1kHz & 1.117 & 0.043 & 140,1 \\
				\hline
				5kHz & 1.122 & 0.952 & 135.0 \\
				\hline
				9kHz & 1.121 & 1.864 & 28.6 \\
				\hline
				\multicolumn{4}{|c|}{Node 4:   } \\
				\hline
				1kHz & 1.117 & 0.422 & 37.3 \\
				\hline
				5kHz & 1.122 & 0.493 & 43.9 \\
				\hline
				9kHz & 1.121 & 1.302 & 13.6 \\
				\hline
			\end{tabular}
		}
		\end{center}
		\label{tab:A}
		\caption{evaluation board measurements for Circuit A}
	\end{table}
	\newpage
	\subsection{Circuit B}
		\begin{figure}[!ht] %circuit B
					\begin{center}
				\begin{circuitikz}[scale = 0.75, transform shape]
					\draw (0,0)
					node[bnc](B){CON2} to[short](2,0)
					node[above]{1}to[short,*-,](2,0)
					to[C, l=$C_{22}$, a=100nF](4,0)
					node[right]{3}to[short,-*](4,0)
					;
					\node[ground] at (B.shield){};
					\draw 
					(0,3) node[ground]{}
					to[R, l=$R_{21}$, a=1k$\Omega$](4,3)
					node[above]{2}to[short, -*](4,3)
					;
					\draw 
					(0,-3) node[ground]{}
					to[C, l=$C_{23}$, a=100nF](4,-3)
					node[below]{4}to[short, -*](4,-3)
					;
					\draw 
					(4,3) -- (7,3)
					to[R, l=$R_{22}$, a=1k$\Omega$](7,-3) -- (4,-3)
					to[L, l=$L_{21}$, a=10mH](4,0)
					to[C, l=$C_{21}$, a=47nF](4,3)
					;
				\end{circuitikz}
				\label{fig:B}
				\caption{circuit B}
			\end{center}
		\end{figure}
				\begin{table}[!ht] %table circuit B
				\begin{center}
					\resizebox{75mm}{!}{
					\begin{tabular}{|c|c|c|c|}
						\hline
						\multicolumn{4}{|c|}{\textbf{Circuit B:}} \\
						\hline\hline
						Freq [kHz]: & Channel 1 [V]: & Channel 2 [V]: & Angle[°]: \\
						\hline
						\multicolumn{4}{|c|}{Node 1:   } \\
						\hline
						1kHz &   &   &   \\
						\hline
						5kHz &   &   &   \\
						\hline
						9kHz &   &   &   \\
						\hline
						\multicolumn{4}{|c|}{Node 2:   } \\
						\hline
						1kHz & 1.117 & 0.250 & 28.1 \\
						\hline
						5kHz & 1.122 & 0.245 & -42.9 \\
						\hline
						9kHz & 1.121 & 0.921 & 68.0 \\
						\hline
						\multicolumn{4}{|c|}{Node 3:   } \\
						\hline
						1kHz & 1.117 & 0.486 & 22.6 \\
						\hline
						5kHz & 1.122 & 0.332 & 69.0 \\
						\hline
						9kHz & 1.121 & 1.279 & 24.0 \\
						\hline
						\multicolumn{4}{|c|}{Node 4:   } \\
						\hline
						1kHz & 1.117 & 0.502 & 20.6 \\
						\hline
						5kHz & 1.122 & 1.077 & -13.5 \\
						\hline
						9kHz & 1.121 & 0.503 & -114.5 \\
						\hline
				\end{tabular}
			}
			\end{center}
			\label{tab:B}
			\caption{evaluation board measurements for Circuit B}
		\end{table}
	\section{summery}
	
\end{document}