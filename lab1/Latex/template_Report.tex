\documentclass[notitlepage, a4paper, 11pt]{article}

\usepackage{geometry}
\geometry{
	a4paper,
	total={170mm,257mm},
	left=20mm,
	top=20mm,
}

\usepackage{graphicx}
\usepackage{amsmath}
\usepackage{listings}
\usepackage{xcolor}
\usepackage{minted}
\usepackage{tikz}
\usepackage[european resistors]{circuitikz}
\usepackage{caption}
\usepackage{subcaption}
\usepackage{hyperref}
\hypersetup{
	pdfborder = false,
	colorlinks=true,
	linkcolor=black,
	filecolor=black,      
	urlcolor=black,
	pdftitle={Overleaf Example},
	pdfpagemode=FullScreen,
}
\title{Nodal Analysis\\
	\large Laboratory I}
\author{Patrycja Nazim, Adrian Król, Kamil Chaj}
\date{}

\begin{document}
	\maketitle
	\section{Aim of exercise}
	The aim of our exercise was to experimentally verify the nodal analysis in RLC circuits. We have achieved it by measuring the voltages on different nodes of the chosen circuits using a dedicated evaluation board and vector voltmeter. The obtained measurement results are compared with analytical calculations.
	
	Apart from the values of potentials in individual nodes of the circuits being measured, we calculated the currents flowing through pointed elements.

	\section{Course of measurements}\label{sec:course-of-measurements}
	First step of our measurements was connecting evaluation board(fig.\ref{fig:ms}) to laboratory computer via USB and starting Vector Voltage Meter software. Next we connected both INPUT1 and CON1 to OUTPUT using BNC cables and splitter, last we connected oscilloscope probe to INPUT2 connector. Now with all preparation finished we started voltage measurements of each node in Circuit A(fig.\ref{fig:A}). Then we disconnected BNC cable connected to CON1 and connected it to CON2 and repeated the same measurements for Circuit B(fig.\ref{fig:B})
	
	\begin{figure}[!ht]
		\begin{center}
			\begin{circuitikz}
				\draw(0, 0)node[bnc](I1){INPUT1};
				\draw [black, ->](I1.hot) -- (3, 0) -- (3, -0.5);
				\draw(3, -0.57)node[right]{node} to[short, -*](3, -0.57);
				\draw(0, -1)node[bnc](I2){INPUT2};
				\draw [black, ->](I2.hot) -- (-1.5, -1) -- (-1.5, -2);
				\draw(0, -2)node[bnc](O){OUTPUT};
				\draw(3, -2)node[bnc, rotate=180](D) {};
				\draw(-3,-2)node[ground] {}
				to[vsourcesin](-1.5,-2)
				to[short](O.hot);
				\draw [black, ->](O.hot) -- (D.hot);
				\draw [black, thick](-4,1) rectangle (0.9, -3);
				\draw [black, thick](2,1) rectangle (4, -3);
				\draw(-1.5, 1.2) node[short]{EVALUATION BOARD} (-1.5, 1.2);
				\draw(3, 1.2) node[short]{DUT} (3, 1.2);
				\draw(3, 0.2) node[short]{probe} (3, 0.2);
				\draw(3, -2.5) node[short]{CON1/2} (3, -2.5);
			\end{circuitikz}
		\end{center}
		\caption{measurements schematic}
		\label{fig:ms}
	\end{figure}

	\section{Nodal analysis - method}\label{sec:Nodal analysis}
	Method which we are going to use to solve these circuits is known as "Nodal Analysis by Inspection". In this method we need to construct 3 matrices: $\mathbf{i}$ - current vector, $\mathbf{u}$ - voltage vector(unknown), $\mathbf{G}$ - conductance matrix, with sizes respectively $\mathit{N} \times 1$, $\mathit{N} \times 1$, $\mathit{N}-1 \times \mathit{N}-1$.
	
	Size of conductance matrix is always smaller then number of nodes in the circuit by one because in nodal analysis one node is connected to ground as reference node.

		\begin{center} $\mathbf{Gu=i}$ \end{center}
		\begin{center}
			\begin{math}
				\begin{bmatrix}
					G_{11} & -G_{12} & -G_{13} \\
					-G_{21} & G_{22} & -G_{23} \\
					-G_{31} & -G_{32} & G_{33} 
				\end{bmatrix}
				\begin{bmatrix}
					U_1 \\
					U_2 \\ 
					U_3
				\end{bmatrix}
				=
				\begin{bmatrix}
					I_1 \\
					I_2 \\
					I_3
				\end{bmatrix}
			\end{math}
		\end{center}
		Where $G_{11}$, $G_{22}$, $G_{33}$ are sums of conductance of each branch connected to the node \newline $G_{12} = G_{21}$, $G_{13} = G_{31}$, $G_{32} = G_{23}$ are sums of conductance of branches between nodes \newline $I_1, I_2, I_3$ are sums of current sources entering or exiting node and $U_1, U_2, U_3$ are unknown voltages that we are trying to find.
		\newline\newline
		With simple matrix operation we obtain equation
		$$
		\mathbf{u} = \mathbf{G}^{-1}\mathbf{i}
		$$
		which can be easily calculated
	\section{Theoretical calculations}
	All calculation are made in Python with NumPy library. Source code for all calculation can be found in the Appendix
	
	Before starting any calculation we modified original circuits' schematics by replacing BNC connectors with AC voltage source and drawing connection between ground(fig. \ref{fig:tAbt}, fig. \ref{fig:tB})
	
	for both circuits amplitude of Voltage source is 0.192V and calculations are made for frequencies 1kHz, 5kHz and 9kHz 
	\subsection{Circuit A}
	In our calculations voltages in nodes 1, 3 and 4 are named $U_1, U_3, U_4$ and reference node is the same node that was ground in original circuit schematic. Lone voltage source must be transfigured in order to solve this circuit by nodal analysis, after source transfiguration circuit looks like this fig. \ref{fig:tAt}
	
		\begin{figure}[!ht] %circuit A
		\centering
				\begin{subfigure}{0.45\textwidth}
					\centering
						\begin{circuitikz}[scale = 0.7, transform shape]
						\draw 
						(2,-2) to[vsourcesin, l=V] (2,0)
						node[left]{$U_1$}to[short, *-](2,2)
						to[R, l=$R_{11}$, a=1k$\Omega$] (6,2)
						node[above]{2}to[short, *-](6,2)
						to[C, l=$C_{11}$, a=100nF] (10,2) -- (10,0)
						node[right]{$U_4$}to[short,*-](10,0)
						to[R, l=$R_{12}$, a=1k$\Omega$] (6,0)
						node[above]{$U_3$}to[short, -*](6,0)
						to[C, l=$C_{12}$, a=47nF, i<=$I_{C_{12}}$] (2,0)
						;
						\draw 
						(6,0)
						to[L, l=$L_{11}$, a=10mH](6,-2)
						;
						\draw 
						(10,0) to[C, l=$C_{13}$, a=22nF](10,-2)
						;
						\draw (2,-2)
						to[short](10,-2);
						\draw (6,-2)
						node[rground] {} (6,-2);
					\end{circuitikz}
					\caption{}
					\label{fig:tAbt}
				\end{subfigure}
				\begin{subfigure}{0.45\textwidth}
					\centering
						\begin{circuitikz}[scale = 0.7, transform shape]
							\draw 
							(2,-2) to[short] (2,0)
							node[left]{$U_1$}to[short, *-](2,0)
							to[vsourcesin, l=V](2,2)
							to[R, l=$R_{11}$, a=1k$\Omega$] (6,2)
							node[above]{2}to[short, *-](6,2)
							to[C, l=$C_{11}$, a=100nF] (10,2) -- (10,0)
							node[right]{$U_4$}to[short,*-](10,0)
							to[R, l=$R_{12}$, a=1k$\Omega$] (6,0)
							node[above]{$U_3$}to[short, -*](6,0)
							to[C, l=$C_{12}$, a=47nF, i<=$I_{C_{12}}$](3,0)
							to[vsourcesin, l=V](2,0)
							;
							\draw 
							(6,0)
							to[L, l=$L_{11}$, a=10mH](6,-2)
							;
							\draw 
							(10,0) to[C, l=$C_{13}$, a=22nF](10,-2)
							;
							\draw (2,-2)
							to[short](10,-2);
							\draw (6,-2)
							node[rground] {} (6,-2);
						\end{circuitikz}
						\caption{}
						\label{fig:tAt}
				\end{subfigure}
			\caption{theoretical circuit A}
			\label{fig:tA}
	\end{figure}

	following rules of Nodal Analysis by Inspection(Sec. \ref{sec:Nodal analysis}) we obtain $\mathbf{Gu=i}$ equation
	
	$$
		\centering
		\begin{bmatrix}
			G_{R_{11}+C_{11}} + G_{C_{12}} & -G_{C_{12}} & -G_{R_{11}+C_{11}}\\[6pt]
			-G_{C_{12}} & G_{C_{12}} + G_{L_{11}} + G_{R_{12}} & -G_{R_{12}}\\[6pt]
			-G_{R_{11}}-G_{C_{11}} & -G_{C_{12}} & G_{R_{11}+C_{11}} + G_{R_{12}} + G_{C_{12}}
		\end{bmatrix}
		\begin{bmatrix}
			U_1\\[6pt]
			U_3\\[6pt]
			U_4
		\end{bmatrix}
		=
		\begin{bmatrix}
			\frac{-V}{Z_{R_{11}+C_{11}}}+\frac{-V}{Z_{C_12}}\\[6pt]
			\frac{V}{Z_{C_12}}\\[6pt]
			\frac{V}{Z_{R_{11}+C_{11}}}
		\end{bmatrix}
	$$
	
	after solving above equation we can easily calculate current of the capacitor $C_{12}$
	
	$$I_{C_{12}} = \frac{U_3 - U_2}{Z_{C_{12}}}$$

	all that is left to do is computing	
	\subsubsection*{results of calculations}
	source code in the Appendix. \ref{code:A}
	\begin{center}
		!!!WRONG RESULTS... I THINK!!!
		\resizebox{165mm}{!}{
			\begin{tabular}{lrr}
				\hline
				1000.0 Hz   &   Magnitude &    Phase \\
				\hline
				Node 1      & 0.192       & 180      \\
				Node 3      & 1.22663e-18 & 135      \\
				Node 4      & 0           &   0      \\
				I of C\_12   & 5.66995e-05 &  -1.5708 \\
				\hline
			\end{tabular}
			\begin{tabular}{lrr}
				\hline
				5000.0 Hz   &   Magnitude &     Phase \\
				\hline
				Node 1      & 0.192       & -180      \\
				Node 3      & 2.86098e-17 &  104.036  \\
				Node 4      & 1.43049e-17 &   14.0362 \\
				I of C\_12   & 0.000283497 &   -1.5708 \\
				\hline
			\end{tabular}
			\begin{tabular}{lrr}
				\hline
				9000.0 Hz   &   Magnitude &     Phase \\
				\hline
				Node 1      & 0.192       &  180      \\
				Node 3      & 5.72196e-17 & -165.964  \\
				Node 4      & 2.08167e-17 &  -90      \\
				I of C\_12   & 0.000510295 &   -1.5708 \\
				\hline
			\end{tabular}
	}
	\end{center}

	\subsection{Circuit B}
	Following similar steps as in Circuit A, voltages in notes 2, 3 and 4 are named $U_2, U_3, U_4$ and reference node is the same node that was ground in original circuit schematic. There is no lone voltage source so source transfiguration is not necessary.
	
		\begin{figure}[!ht] %circuit B
		\begin{center}
			\begin{circuitikz}[scale = 0.75, transform shape]
				\draw (0,0)
				to[vsourcesin, l=V](2,0)
				node[above]{1}to[short,*-,](2,0)
				to[C, l=$C_{22}$, a=100nF](4,0)
				node[right]{$U_3$}to[short,-*](4,0)
				;
				\draw (0,3)
				to[R, l=$R_{21}$, a=1k$\Omega$](4,3)
				node[above]{$U_2$}to[short, -*](4,3)
				;
				\draw (0,-3)
				to[C, l=$C_{23}$, a=100nF](4,-3)
				node[below]{$U_4$}to[short, -*](4,-3)
				;
				\draw 
				(4,3) -- (7,3)
				to[R, l=$R_{22}$, a=1k$\Omega$, i=$I_{R_{22}}$](7,-3) -- (4,-3)
				to[L, l=$L_{21}$, a=10mH](4,0)
				to[C, l=$C_{21}$, a=47nF](4,3)
				;
				\draw (0,3)
				to[short](0,-3);
				\draw (0,0)
				node[rground, rotate=-90] {} (0,0);
			\end{circuitikz}
			\caption{theoretical circuit B}
			\label{fig:tB}
		\end{center}
	\end{figure}
	
	following rules of Nodal Analysis by Inspection(Sec. \ref{sec:Nodal analysis}) we obtain $\mathbf{Gu=i}$ equation
	
	$$
	\begin{bmatrix}
		G_{R_{21}}+G_{R_{22}}+G_{C_{21}} & -G_{C_{21}} &-G_{R_{22}}\\[6pt]
		-G_{C_{21}} & G_{C_{21}} + G_{C_{22}} + G_{L_{21}} & G_{L_{21}}\\[6pt]
		-G_{R_{22}} & -G_{L_{21}} & G_{R_{22}} + G_{C_{23}} + G_{L_{21}}
	\end{bmatrix}
	\begin{bmatrix}
		U_2\\[6pt]
		U_3\\[6pt]
		U_4
	\end{bmatrix}
	=
	\begin{bmatrix}
		0\\[6pt]
		\frac{V}{Z_{C_{22}}}\\[6pt]
		0
	\end{bmatrix}
	$$
	
	after solving above equation we can easily calculate current of the capacitor $R_{22}$
	
	$$
	I_{R_{22}}=\frac{U_4-U_2}{Z_{R_{22}}}
	$$
	
	and these are the results
	\subsubsection{Results of calculations}
	Source code in the Appendix \ref{code:B}
	\begin{center}
		\resizebox{165mm}{!}{
		\begin{tabular}{lrr}
			\hline
			1000.0 Hz   &   Magnitude &    Phase \\
			\hline
			Node 2      & 0.0442579   & 28.1037  \\
			Node 3      & 0.0837746   & 22.3054  \\
			Node 4      & 0.0867875   & 20.4605  \\
			I of R\_22   & 4.33246e-05 & -2.92078 \\
			\hline
		\end{tabular}
		\begin{tabular}{lrr}
			\hline
			5000.0 Hz   &   Magnitude &     Phase \\
			\hline
			Node 2      & 0.0521098   & -53.9736  \\
			Node 3      & 0.0534524   &  86.6776  \\
			Node 4      & 0.206963    & -12.1633  \\
			I of R\_22   & 0.000171674 &   3.13307 \\
			\hline
		\end{tabular}
		\begin{tabular}{lrr}
			\hline
			9000.0 Hz   &   Magnitude &      Phase \\
			\hline
			Node 2      & 0.16282     &   63.622   \\
			Node 3      & 0.219774    &   20.3532  \\
			Node 4      & 0.0750914   & -122.05    \\
			I of R\_22   & 0.000237659 &    1.07918 \\
			\hline
		\end{tabular}	
		}
	\end{center}
	\newpage
	\section{Real measurements}
	Following steps of measurements described in Section \ref{sec:course-of-measurements} we obtain measurements for each circuit
	\subsection{Circuit A}
			\begin{figure}[!ht] %circuit A
			\begin{center}
				\begin{circuitikz}[scale = 0.75, transform shape]
					\draw 
					(1,0) node[bnc](B){CON2} to[short](2,0)
					node[below]{1}to[short, *-](2,2)
					to[R, l=$R_{11}$, a=1k$\Omega$] (6,2)
					node[above]{2}to[short, *-](6,2)
					to[C, l=$C_{11}$, a=100nF] (10,2) -- (10,0)
					node[left]{4}to[short,*-](10,0)
					to[R, l=$R_{12}$, a=1k$\Omega$] (6,0)
					node[above]{3}to[short, -*](6,0)
					to[C, l=$C_{12}$, a=47nF, i<=$I_{C_{12}}$] (2,0)
					;
					\node[ground] at (B.shield){};
					\draw 
					(6,0)
					to[L, l=$L_{11}$, a=10mH](6,-2)
					to[short] node[ground] {} (6,-2)
					;
					\draw 
					(10,0) to[C, l=$C_{13}$, a=22nF](10,-2)
					to[short] node[ground] {} (10,-2)
					;
				\end{circuitikz}
				\caption{circuit A}
				\label{fig:A}
			\end{center}
		\end{figure}
	
		Current $I_{C_{12}}$ is measured indirectly using measured voltage and impedance of the component
		
		$$
		I_{C_{12}}=\frac{U_1-U_3}{Z_{C_{12}}}
		$$
		
		where $U_3$ is voltage in node 3, $U_1$ is voltage in node 1 and $Z_{C_{12}}$ is impedance of capacitor $C_{12}$
	
		\begin{table}[!ht] %table circuit A
		\begin{subtable}{0.6\textwidth}
			\centering
				\resizebox{75mm}{!}{
				\begin{tabular}{|c|c|c|c|}
					\hline
					\multicolumn{4}{|c|}{Circuit A} \\
					\hline
					Freq & Channel 1[V] & Channel 2 [V] & Angle [°] \\
					\hline
					\multicolumn{4}{|c|}{node 1} \\
					\hline
					1kHz & 0.192 & 0.192 & 0 \\
					\hline
					5kHz & 0.192 & 0.192 & -0.1 \\
					\hline
					9kHz & 0.192 & 0.192 & -0.1 \\
					\hline
					\multicolumn{4}{|c|}{node 2} \\
					\hline
					1kHz & 0.192 & 0.145 & -19.3 \\
					\hline
					5kHz & 0.192 & 0.59 & 9.3 \\
					\hline
					9kHz & 0.192 & 0.236 & -9.7 \\
					\hline
					\multicolumn{4}{|c|}{node 3} \\
					\hline
					1kHz & 0.192 & 0.008 & 140.7 \\
					\hline
					5kHz & 0.192 & 0.154 & 136.1 \\
					\hline
					9kHz & 0.192 & 0.376 & 30 \\
					\hline
					\multicolumn{4}{|c|}{node 4} \\
					\hline
					1kHz & 0.192 & 0.072 & 38.6 \\
					\hline
					5kHz & 0.192 & 0.082 & 42.1 \\
					\hline
					9kHz & 0.192 & 0.231 & -11.8 \\
					\hline
				\end{tabular}
			}
			\caption{}
		\end{subtable}
		\begin{subtable}{0.25\textwidth}
			\centering
				\begin{tabular}{|c|c|c|}
				\hline
				freq & $I_{C_{12}}$[I] & angle[°]\\
				\hline
				1kHz &  0.198 & 178.5°\\
				\hline
				5kHz &  $7.543\cdot10^{-5}$ & 1.5°\\
				\hline
				9kHz &  $9.954\cdot10^{-5}$ & -36.9°\\
				\hline
			\end{tabular}
			\caption{}
			\label{subtab:A}
		\end{subtable}
		\caption{evaluation board measurements for Circuit A}
		\label{tab:A}
	\end{table}

	\newpage
	\subsection{Circuit B}
		\begin{figure}[!ht] %circuit B
			\begin{center}
				\begin{circuitikz}[scale = 0.75, transform shape]
					\draw (0,0)
					node[bnc](B){CON2} to[short](2,0)
					node[above]{1}to[short,*-,](2,0)
					to[C, l=$C_{22}$, a=100nF](4,0)
					node[right]{3}to[short,-*](4,0)
					;
					\node[ground] at (B.shield){};
					\draw 
					(0,3) node[ground]{}
					to[R, l=$R_{21}$, a=1k$\Omega$](4,3)
					node[above]{2}to[short, -*](4,3)
					;
					\draw 
					(0,-3) node[ground]{}
					to[C, l=$C_{23}$, a=100nF](4,-3)
					node[below]{4}to[short, -*](4,-3)
					;
					\draw 
					(4,3) -- (7,3)
					to[R, l=$R_{22}$, a=1k$\Omega$, i=$I_{R_{22}}$](7,-3) -- (4,-3)
					to[L, l=$L_{21}$, a=10mH](4,0)
					to[C, l=$C_{21}$, a=47nF](4,3)
					;
				\end{circuitikz}
				\caption{circuit B}
				\label{fig:B}
			\end{center}
		\end{figure}
		Current $I_{R_{22}}$ is measured indirectly using measured voltage and impedance of the component
		
		$$
		I_{R_{22}}=\frac{U_4-U_2}{Z_{R_{22}}}
		$$
		
		where $U_4$ is voltage in node 4, $U_2$ is voltage in node 2 and $Z_{R_{22}}$ is impedance of resistor $R_{22}$
				\begin{table}[ht] %table circuit B		
					\begin{subtable}{0.60\textwidth}
						\centering
						\resizebox{75mm}{!}{
							\begin{tabular}{|c|c|c|c|}
								\hline
								\multicolumn{4}{|c|}{Circuit B} \\
								\hline
								Freq & Channel 1[V] & Channel 2 [V] & Angle [°] \\
								\hline
								\multicolumn{4}{|c|}{node 1} \\
								\hline
								1kHz & 0.192 & 0.192 & 0 \\
								\hline
								5kHz & 0.192 & 0.192 & -0.1 \\
								\hline
								9kHz & 0.192 & 0.192 & 0 \\
								\hline
								\multicolumn{4}{|c|}{node 2} \\
								\hline
								1kHz & 0.192 & 0.044 & 28.7 \\
								\hline
								5kHz & 0.192 & 0.044 & -35.7 \\
								\hline
								9kHz & 0.192 & 0.156 & 69.7 \\
								\hline
								\multicolumn{4}{|c|}{node 3} \\
								\hline
								1kHz & 0.192 & 0.084 & 23.3 \\
								\hline
								5kHz & 0.192 & 0.055 & 62.4 \\
								\hline
								9kHz & 0.192 & 0.22 & 24.4 \\
								\hline
								\multicolumn{4}{|c|}{node 4} \\
								\hline
								1kHz & 0.192 & 0.086 & 21.2 \\
								\hline
								5kHz & 0.192 & 0.18 & -11.8 \\
								\hline
								9kHz & 0.192 & 0.09 & -113.8 \\
								\hline
							\end{tabular}
						}
						\caption{}
					\end{subtable}
					\begin{subtable}{0.25\textwidth}
						\centering
						\begin{tabular}{|c|c|c|}
							\hline
							freq & $I_{R_{22}}$[I] & Angle[°]\\
							\hline
							1kHz & 4.379$\cdot10^{-5}$ & 28.83 \\
							\hline
							5kHz & 1.407$\cdot10^{-4}$ & 175.48 \\
							\hline
							9kHz & 2.456$\cdot10^{-4}$ & 68.44\\
							\hline
						\end{tabular}
						\caption{}
						\label{subtab:B}
					\end{subtable}
					\label{tab:B}
			\caption{evaluation board measurements for Circuit B}
		\end{table}
	\newpage
	\section{Comparison}\label{sec:comparison}
	Results of measurements are in general similar to results of nodal analysis. In most cases at least first significant digit of amplitude is the same in both results and phase shift is within few percent of error, but with increasing frequency calculations become less precise. For some applications first significant digit precision can be not satisfying, but roughly both results are comparable. In order to verify how precise nodal analysis is, more data and more precise measuring equipment are needed
	\section{Summery}
	Nodal analysis broadly speaking is good method to evaluate values in a circuit, real simple for circuits with many branches and very easy to compute with computer aid because. But in terms of precision as stated in Section \ref{sec:comparison} for some applications evaluation can be not good enough. 
	\newpage
	\appendix
	\section*{Appendix}\label{Appendix}
	\href{https://github.com/kamilix2003/CT-labs}{GITHUB repository}
	\section{Source code circuit A}
	\inputminted{python}{../CircuitA.py}
	\label{code:A}
	\newpage
	\section{Source code circuit B}
	\inputminted{python}{../CircuitB.py}
	\label{code:B}
\end{document}